The human brain is a system of neurons that provides great computational capacity via a complicated communication network. A complete description of this network, also known as the human connectome, is an active topic of research as it would provide invaluable knowledge to cognitive neuroscience. Historically, the study of brain connectivity has been marked by a dichotomy between structure and function. Recent advances in magnetic resonance imaging (MRI) provide a potential framework for bridging this divide and allowing for a more complete window onto brain connectivity. 

Here, framework for examining structural connectivity and functional connectivity in a network is presented with a focus upon the study of well-known sub-networks. An examination of the thalamo-cortical system exhibits the ability to use multivariate MRI data to model a brain network and identify localized connectivity disruptions. A study of a language-based decision-making task shows how functional and structural connectivity may be used to examine the integration of multiple subnetwork and their influence upon cognitive performance. Finally, a study of lateralization in aginng demonstrates both functional and structural changes with aging are associated with maintaining a high level of cognitive performance.
