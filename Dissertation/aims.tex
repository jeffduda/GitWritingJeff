% Specific Aims
% 1-2 pages: describe overall purpose of project, list individual goals
%
\chapter{Specific Aims}
\label{chap__aims}

\section{Development of a Graph-Based Framework for Examining Connectivity}
Connectivity between cortical regions in the brain is essential for proper functioning and any disruption in white matter represents disruption of the computational network. Here, these networks will be represented with a graph-based model where nodes represent cortical regions and edges represent the white matter fiber bundles that connect these functional regions. A focus of this work will be using diffusion tensor MRI and fiber tractography to quantitatively examine connectivity between cortical regions to define edge-weights in this graph-based model.  

\begin{itemize}
\item
Quantitative streamline based metrics will be developed for identifying pathology in structural connectivity as well as for characterizing connectivity phenotypes.

\item
A limiting property of current tractography methods is their inability to identify tracts in the presence of a white matter disruption. In order to overcome this, methods for adapting atlas-based tractography to subject-specific data will be developed.


\item
A graph-based model will be used to represent functional networks in the brain. Structural connectivity metrics will be used to define edge-weights in order to provide a framework for comparison between populations.

\end{itemize}

\section{Evaluating the Connectivity Framework}
Structural connectivity metrics will be examined in a healthy control atlas to quantify connectivity in known networks. Patient populations with known connectivity pathologies will provide a further characterization of the tract-based metrics in projection, commissural and associative white matter fiber tracts.
\begin{itemize}

\item
The prefrontal fibers of the thalamo-cortical network will be examined in a traumatic brain injury population to demonstate the framework's ability to identify localized connectivity disruptions.

\item
Commissural fibers of the human corpus callosum will be examined to detect connectivity phenotypes in Alzeheimer's disease and corticobasal degeneration, populations with known connectivity deficiencies. 

\item
A multivariate atlas will be used to fully charactize the metrics for quantifying structural connectivity in white matter in healthy young and elderly control population as well as in patient populations.

\end{itemize}

\section{Demonstrating Clinical Utility of the Connectivity Framework}
The language network is well known to be implicated in a number neurological disorders and  will be used to explore connectivity differences.

\begin{itemize}

\item
A multivariate model of the language network will be constucted in order to provide a framework for examining both functional and structural connectivity in healthy controls. The relationship between functional connectivity and structural connectivity will be explicitly characterized using language task functional data, cortical thickness measures and white matter connectivity measures.

\item
The language network will be examined in multiple patient populations with known connectivity disorders in order to gain deeper insight into common aspects of connectivity degradation as well as elucidate unique connectivity phenotypes.

\item
In order to make the tools more useful to the scientific community, the methods will be implemented as an open source toolkit. Routines allowing for the integration of additional open source tools, such as Camino and Advanced Normalization Tools (ANTs) will also be provided.

\end{itemize}


